% chktex-file 18
% 
% Hinterlegen Sie basierend auf Ihrem Vorgehen einen ersten Projektplan / Gant Chart
% und geben Sie an, inwiefern dieser mit weiteren Partnern abgesprochen ist.
\addcontentsline{toc}{section}{\normalfont{Project Plan}}
\section*{Project Plan}

\begingroup
\def\arraystretch{2}
\setlength\tabcolsep{20pt}

\begin{tabular}{ rl }

\AdvanceDate[6]\today & Submit Exposé \\
\AdvanceDate[7]\today & Register thesis? \\
\AdvanceDate[0] & Oberthemen identifizieren \\
\AdvanceDate[3] & Oberthemen Keywords sammeln \\
\AdvanceDate[2] & Ansichtsweisen identifizieren \\
\AdvanceDate[4] & Alle Perspektiven listen und erforschen \\
\AdvanceDate[7] & Meinungen Clustern \\
\AdvanceDate[25] & Verbindungen und Abhängigkeiten herstellen \\
\AdvanceDate[28] & Thesen finden auf Grund der Recherche \\
\AdvanceDate[31] & Erste Kapitel schreiben \\
\AdvanceDate[34] & Fragen finden und sammeln \\
\AdvanceDate[38] & Fragen auswählen \\
\AdvanceDate[42] & Kandidaten finden \\
\AdvanceDate[43] & Kandidaten Auswahl \\
\AdvanceDate[7] & Fragen Katalog zusammenstellen \\
\AdvanceDate[7] & Instanz zum auswerten finden und aufsetzen \\
\AdvanceDate[7] & Kandidaten für Einzelgespräche oder Remoteausführung einladen \\
\AdvanceDate[7] & Datensammeln und ggf. Fragen nachschärfen \\
\AdvanceDate[7] & Erste Auswertung um einzuschätzen ob genug Daten erhoben worden sind \\
\AdvanceDate[7] & Klare Ergebnisse filtern \\
\AdvanceDate[7] & Indirekte Abhängigkeiten identifizieren \\
\AdvanceDate[7] & Zusammenhänge mit den Thesen erschließen \\
\AdvanceDate[7] & Weitere Kapitel schreiben \\
\AdvanceDate[7] & Fazit schreiben \\
\AdvanceDate[7] & Kandidaten informieren \\
\AdvanceDate[7] & Einleitung schreiben \\
\AdvanceDate[7] & Next Steps nach der Arbeit identifizieren und listen \\
\AdvanceDate[7] & Korrektur lesen \\
\AdvanceDate[7] & Präsentation anfertigen \\
\AdvanceDate[97]\today & Thesis submission \\
\AdvanceDate[100]\today & Colloquium \\
\end{tabular}

\endgroup