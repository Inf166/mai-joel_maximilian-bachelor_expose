% chktex-file 18
% chktex-file 8
% Schildern Sie, warum das Thema relevant ist.
% Beschreiben Sie dann die konkrete Problemstellung, die Ihre Arbeit untersuchen wird.
\addcontentsline{toc}{section}{\normalfont{Motivation and problems to solve}}
\section*{Motivation and problems to solve}

% Problems:
% 
% - Waterfall processes come with risks for companies because they have to be sure they got all requirements sorted out
%   - If not, they can not react fast enough to changes in requirements and therefore lose money and time.
%     - But it is almost given, that changes will come during development
%       - Therefore those products will more often than not fail in the real market
% - All members of the project and the Client had to commit to the agile manifest
%   - A big share of all Web Agencies kept parts of the original waterfall phases in their processes
%     - Developers and Designers can not fully work with agile
%       - Only the negative Impact of agile is shown in those agencies
% 
% - The project needs to be desirable to the end-user
% - The project needs to be technically feasible
% - The project needs to deliver business value fast
% 
% - Improvements to Development or Design phases kept their focus on each separate phase in dictating the other
% - Agile problems
% - For example Agile wants Designers to work in smaller increments, but they usually want to build a finished product knowing all requirements
%   - Instead of collaborating more, these improvements result in less collaboration
%     - Both Teams work after each other with a slight head start 
%     - Instead of working parallel or intertwined    
%   - Designers are driven by deadlines and collaborate less with developers, which leads to not viable solutions 
%   - Developers build working software, which is no longer valuable to the end-user
% - Human-centered Design problems
%   - Human-centered instead of User-centered
%   - keeps the end-user in mind till a finished product concept is reached
%     - Write Business Stories instead of User Stories
%       - The way of writing a user story can prohibit the actual solution that may be more feasible
%     - During the Development of said concept, the end-user requirements are forgotten and replaced by technical restrictions
% - Quality Assurance Department Problems
%   - the final product gets deployed and only receives irregular updates
%   - neither agile nor a sustainable view of digital products
%   - Requirements will change, therefore the way of reaching a solution must too
%
% - Differentiate between starting a new project and maintaining a product problems
%   - Starting a new project
%     - Software Architects are not included in concept phases
%     - Designers waste time on fetched-out designs, and development has to wait until the design is set
%   - Maintaining a product
%     - Designers and Developers do not collaborate closely 
%     - They do not discuss feasible solutions together with the client (project management does with no information on feasibility) 
% 
% - Fixing all of the problems that remain
%   - We still need to keep in mind the time, budget, and scope of the projects
%     - subgoal should also be to minimize waste and unused documentation
% 
% 
% A project needs to be desirable by the end-user, technically feasible, and deliver fast business value,\citep{ratcliffe2011agile:11}. To solve this problem, the agile movement came up with many ideas, such as splitting the workload into smaller chunks, to tackle these tasks \citep{Theartof49:online}. Integrating the agile manifest meant, that all members of the project and the Client had to commit to it \citep{ratcliffe2011agile:11}. Therefore a big share of all Web Agencies kept parts of the original waterfall phases in their processes \citep{10spanne26:online}. This led to stiff deadlines which could not be changed flexibly by reacting to changes in requirements by an agile team of developers and designers \citep{ratcliffe2011agile:4}.
% 
% Additionally, there is a different strategy out there, that claims to solve similar problems such as agile. The process of Human-centered Design is used to create a feasible design concept. \newline
% More often than not, the end-user is only kept in mind during early concepts. After the solutions are sketched, wireframed, or mocked up there is little to no discussion if the final product pleases the end-user \citep{ratcliffe2011agile:33}. HCI methods, like extended user research by Interviews or by creating concept/navigation maps and testing those on users, keep the end-user in mind and focus on perfecting the Interaction between the user and the computer or system, as the name suggests. 
% 
% So to wrap up, the waterfall phases are highly outdated and come with huge risks for the company, like the stiff development phase with no way of reacting to changing requirements, which results in unusable, unwanted products that fail in the real market \citep{ratcliffe2011agile:14}. Some of the phases were improved. Such as iterative design processes and agile development. But both improvements only put focus on their phase and dictate the others \citep{ratcliffe2011agile:22}. Therefore creates new problems, like less collaboration between designers and developers, or less frequent business value delivered, because both processes are still more often than not after each other and neither intertwined nor parallel to each other.
% 
% To improve this process and to archive the goal, methods from HCI are needed and have to be integrated and the overall process of creating products has to be modified. Another problem is the mentioned little collaboration between Departments \citep{ratcliffe2011agile:fw}. While designers are driven by deadlines and collaborate less with developers, which leads to not viable solutions, developers on the other hand build working software, which is no longer valuable to the end-user \citep{ratcliffe2011agile:19,ratcliffe2011agile:33}. 
% 
% Once the Quality Assurance Department, if there is one, has done its job, the final product gets deployed and only receives irregular updates \citep{ratcliffe2011agile:18}. This is neither agile nor a sustainable view of digital products. Requirements will change, therefore the way of reaching a solution must too \citep{ratcliffe2011agile:30:31}.  
% 
% We still need to keep in mind the time, budget, and scope of projects. But a subgoal should also be to minimize waste and unused documentation. The Goal will be to bring it all together.