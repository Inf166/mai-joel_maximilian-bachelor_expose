% chktex-file 18
% chktex-file 8
% chktex-file 2
% 
% Schildern Sie, warum das Thema relevant ist.
% Beschreiben Sie dann die konkrete Problemstellung, die Ihre Arbeit untersuchen wird.
\addcontentsline{toc}{section}{\normalfont{Motivation and problems to solve}}
\section*{Motivation and problems to solve}

Becoming agile is desired not only by developers anymore. Businesses all over the world try to become agile through many different strategies. Some are more successful than others. But becoming agile seems to be more difficult than thought.

To become agile, one has to first understand what agile is. The word agile stands for flexible business culture. It's about responding to change quickly instead of following a plan. Individuals and Collaboration are more important to agile businesses than processes, methods or documentation. Note the more here. It's not that, agile businesses don't have processes or documentation. Working quality software is just more important. 

To make quality software and not waste any time, the high management has to trust its employees. Agile is a culture based on trust. If an employee has to ask for every change to the live system or to buy needed software tools, a business cannot become agile.

Agile is a culture based on self-organizing teams. If cross-functional teams self-organize, no time is wasted and the single increments can be developed much more quickly. Every piece of software that is developed earlier means, that the company can see the impact of the software more quickly and therefore react to the market and users. Outcome over Output. It's not about putting more out the door. You don't have to release every day. But releasing more often and in small batches, allows for more clear data to test and improve on.

Agile is no divine solution, but a process to become more successful. Often Companies look forward to Agile Coaches or even SCRUM Coaches in the hope of a divine solution that fixes their broken processes or products. But just using the Framework, Processes or Tools does not fix anything. Often these actions lead to "scrumfall", a mix of agile tools and processes that still resemble the waterfall project management structure. To fix this, the whole business has to embrace the agile manifesto and live it.

Since SCRUM is the most popular agile Framework, problems arising with using frameworks will be discussed by looking at problems using SCRUM as an example. SCRUM was developed as a wrapper for eXtreme Programming. It has worked in the companies Jeff Sutherland and Ken Schwaber had integrated them, therefore Sutherland made it a framework and since then many companies have adopted it or changed it up and integrated parts of it into their process. But why is it, that so many businesses fail with scrum, adopt it poorly, or even become less efficient with the new integrated meetings and methods?
