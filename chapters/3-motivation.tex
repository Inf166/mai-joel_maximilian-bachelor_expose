% chktex-file 18
% chktex-file 8
% chktex-file 2
% Schildern Sie, warum das Thema relevant ist.
% Beschreiben Sie dann die konkrete Problemstellung, die Ihre Arbeit untersuchen wird.
\addcontentsline{toc}{section}{\normalfont{Motivation and problems to solve}}
\section*{Motivation and problems to solve}

% Problems:
The Waterfall process comes with risks for companies. To implement such a process correctly, all requirements for the finished product have to be sorted out at the beginning of the project, therefore requiring a lot of time at the start, leaving little time to react to changes. But it is almost standard for software projects to have changing requirements during the development phase, which results in undesirable products for the end-user or unsatisfied clients \cite[p. 14]{ratcliffe2011agile}. The cost of change compounds with time \cite[p. 22]{ratcliffe2011agile}.
% - Waterfall processes come with risks for companies because they have to be sure they got all requirements sorted out
%   - If not, they can not react fast enough to changes in requirements and therefore lose money and time.
%     - But it is almost given, that changes will come during development
%       - Therefore those products will more often than not fail in the real market

Out of these problems arise a few requirements to the product, that has to be met always. The product has to be desirable to the end-user. It has to be technically feasible. And to project has to deliver business value fast.
% - The project needs to be desirable to the end-user
% - The project needs to be technically feasible
% - The project needs to deliver business value fast

To solve these problems, developers and designers came up with solutions, such as working agile or having iterative design phases. Starting with agile, the most frequent problem with agile comes with implementing an agile framework in the company. All members of the project and the client have to commit to the agile manifesto. But most clients cannot leave someone at the agency to stay in contact and collaborate with the teams. Therefore a big share of Web Agencies kept parts of the original waterfall phases in their processes \citep{10spanne26:online}. But because of this management decision, developers and designers cannot fully work agile, and therefore only the negative impacts of working the agile show, such as messy tickets, volatility errors, and opaque workflows. This needs additional tweaking such as the backlog \citep{nakao2014using}.
% - Agile problems
% - All members of the project and the Client had to commit to the agile manifest
%   - A big share of all Web Agencies kept parts of the original waterfall phases in their processes
%     - Developers and Designers can not fully work with agile
%       - Only the negative Impact of agile is shown in those agencies

Improvements to Development or Design phases kept their focus on each separate phase in dictating the other \cite[p. 22]{ratcliffe2011agile}. For example, agile frameworks usually need designers to work in smaller increments \citep{Theartof49:online}, with not enough time to research, which results in concepts that do not grab the whole aspect of all requirements or rely on their assumptions \citep{HowtoCom22:online}. This results often in less collaboration between the two teams \cite[p. iv]{ratcliffe2011agile}\cite[p. 115]{Bringing52:online}. Both teams work after one another instead of parallel or intertwined \cite[p. 9]{Forbrig2015ManagingTA}. Therefore Designers are driven by the deadlines set by the project manager for their concepts and not collaborating closely with software architects leads to unviable solutions \cite[p. 4]{ratcliffe2011agile}. But Developers build working software, which is no longer valuable to the end-user because they put technical feasibility over the set boundaries by the UX team. The lack of focus put on the end-user in agile development is a major concern and the biggest weakness \cite[p. 312]{Humancom0:online}.
% - Improvements to Development or Design phases kept their focus on each separate phase in dictating the other 
% - For example Agile wants Designers to work in smaller increments, but they usually want to build a finished product knowing all requirements
%   - Instead of collaborating more, these improvements result in less collaboration
%     - Both Teams work after each other with a slight head start 
%     - Instead of working parallel or intertwined    
%   - Designers are driven by deadlines and collaborate less with developers, which leads to not viable solutions 
%   - Developers build working software, which is no longer valuable to the end-user

The Human-centered Design approach solves many of the problems above, but also comes with problems. First of all, it is important, that the Design includes all humans in the required research. Concentrating only on the end-user will leave gaps in the requirements that can lead to an undesirable product by the client. More often than not, the Researcher writes User Stories wrong \citep{userstor59:online}. They already have a solution in mind, leaving out the possibility of a better solution. Usually, the User is kept in mind during the development of the concept, by Empathy Maps or other Tools. But once the concept is finished and given to the development team, those end-user requirements are forgotten and replaced by technical restraints, leading again to an undesirable product \cite[p. 19, p. 33]{ratcliffe2011agile}.
% - Human-centered Design problems
%   - Human-centered instead of User-centered
%   - keeps the end-user in mind till a finished product concept is reached
%     - Write Business Stories instead of User Stories
%       - The way of writing a user story can prohibit the actual solution that may be more feasible
%     - During the Development of said concept, the end-user requirements are forgotten and replaced by technical restrictions

Once a product reaches the Quality Assurance Department, other problems show up, like that the final deployed product only receives irregular Updates. Neither agile frameworks nor human-centered design keeps in mind, that requirements after deployment can change, and therefore the view on sustainable products has to change too \cite[p. 18, pp. 30-31]{ratcliffe2011agile}.
% - Quality Assurance Department Problems
%   - the final product gets deployed and only receives irregular updates
%   - neither agile nor a sustainable view of digital products
%   - Requirements will change, therefore the way of reaching a solution must too

It is needed to differentiate between starting a new project and maintaining a product. When starting a new project, software architects are not included during the concept phases, which leads to no feasible solution. Designers usually waste time fetching out documentation and designs, while Developers have to wait until the design is set. When maintaining a product, Designers and Developers need to collaborate more closely. They also have to be in the meetings as Consultants to discuss feasible solutions with the client to develop a satisfying product. The business also has to listen and measure the success of the product in the live environment.
% - Differentiate between starting a new project and maintaining a product problems
%   - Starting a new project
%     - Software Architects are not included in concept phases
%     - Designers waste time on fetched-out designs, and development has to wait until the design is set
%   - Maintaining a product
%     - Designers and Developers do not collaborate closely 
%     - They do not discuss feasible solutions together with the client (project management does with no information on feasibility) 

HCI and Agile share the values of user-focus and iterative cycles, which leads to the conclusion, that combining them and solving the main problems, should lead to a workflow, that solves all key problems \citep{HumanCom2:online}. When trying to fix all of the problems stated above, we still need to keep in mind the time, budget, and scope of the projects. Therefore a subgoal should be to reduce waste and unused documentation.
% - Fixing all of the problems that remain
%   - We still need to keep in mind the time, budget, and scope of the projects
%     - subgoal should also be to minimize waste and unused documentation