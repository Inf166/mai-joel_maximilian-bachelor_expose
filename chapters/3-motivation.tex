% chktex-file 18
% Schildern Sie, warum das Thema relevant ist.
% Beschreiben Sie dann die konkrete Problemstellung, die Ihre Arbeit untersuchen wird.
\addcontentsline{toc}{section}{\normalfont{Motivation and problems to solve}}
\section*{Motivation and problems to solve}

A project needs to deliver fast business value, to be technically feasible and desirable by the end-user \citep{ratcliffe2011agile:11}. To solve this problem, the agile movement came up with many ideas, such as splitting the workload into smaller chunks, to tackle these tasks \citep{Theartof49:online}. But they came at a cost. Integrating the agile manifest meant, that all members of the project and the Client had to commit to it \citep{ratcliffe2011agile:11}. Therefore a big share of all Web Agencies kept parts of the original waterfall phases in their processes \citep{10spanne26:online}. This led to stiff deadlines which had to be met by an agile team of developers and sometimes also designers \citep{ratcliffe2011agile:4}.

Additionally, there is a different strategy out there, that claims to solve similar problems such as agile. Human-computer interaction (HCI) methods to create a feasible design concept are also used. \newline
More often than not, the end-user is only kept in mind during early concepts. After the solutions are sketched, wireframed, or mocked up there is little to no discussion if the final product pleases the end-user \citep{ratcliffe2011agile:33}. HCI methods keep the end-user in mind and focus on perfecting the Interaction between the user and the computer or system, as the name suggests. 

So to wrap up, the waterfall phases are highly outdated and come with huge risks for the company \citep{ratcliffe2011agile:14}. Some of those phases were improved. Such as iterative design processes and agile development. But both only put focus on their phase and dictate others \citep{ratcliffe2011agile:22}. 

To improve this process and to archive the goal, methods from HCI are needed and have to be integrated and the overall process of creating products has to be modified. Another problem is the mentioned little collaboration between Departments \citep{ratcliffe2011agile:fw}. While designers are driven by deadlines and collaborate less with developers, which leads to not viable solutions, developers on the other hand build working software, which is no longer valuable to the end-user \citep{ratcliffe2011agile:19,ratcliffe2011agile:33}. 

Once the Quality Assurance Department, if there is one, has done its job, the final product gets deployed and only receives irregular updates \citep{ratcliffe2011agile:18}. This is neither agile nor a sustainable view of digital products. Requirements will change, therefore the way of reaching a solution must too \citep{ratcliffe2011agile:30:31}.  

We still need to keep in mind the time, budget, and scope of projects. But a subgoal should also be to minimize waste and unused documentation. The Goal will be to bring it all together.