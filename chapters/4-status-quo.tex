% chktex-file 18
% Erläutern Sie in groben Zügen den jetzigen Kenntnisstand zu diesem Problem auf Basis
% der Vorgaben Ihres Industriepartners und / oder auf Basis einer Literaturrecherche
% Für Bachelor- und Masterarbeiten wird eine erste Einordnung in die Literatur erwartet:
% Was sind die wichtigsten Aussagen im Kontext Ihres Problems.
% Bitte zitieren Sie korrekt.
\addcontentsline{toc}{section}{\normalfont{Status Quo}}
\section*{Status Quo}

% Currently, it is State of the art to use agile management methods, such as scrum, lean start-up, or design thinking, to generate business value faster, reduce risks, increase flexibility and improve customer satisfaction. 
% 
% Therefore, there are many publications available that discuss integrating user-centered approaches into agile development \citep{PDFAnAgi97:online, HowtoCom22:online} some even going as far as to call it human-centered \citep{HumanCen72:online, Forbrig2015ManagingTA, Agilehum49:online}. The difference is, that user-centered is end-user focussed, and human-centered means, keeping the focus on all humans interacting with the system.
% 
% As for Human-computer interaction, the most talked-about strategy is human-centered design, which is similar to an agile iterative process, but it focuses on the humans that will use the final product. Agile Experience Design \citep{ratcliffe2011agile:main} is another strategy that tries to improve on the negatives of agile as it is.
% 
% Another approach is to integrate agile into the human-centered design by working one sprint ahead of the design team, or working parallel and closely together, thereby sharing their finished increments and improving on each other. Last but not least there is an approach where the UX Steps of Iteration are completely integrated into the sprint. By working together and not before or aside from the development team, UX becomes an important part of the sprint which results in a product that is not only technically feasible but also usable and has improved user experience \citep{Bringing52:online}.
% 
% Surrounding Human-computer interaction, the most complete set of methods and strategies as well as recommendations can be received by the DIN Standards Committee Ergonomics on their Norms \citep{dinEnIso9241:110, dinEnIso9241:210, dinEnIso9241:220}. More Information on how to write valuable user stories is found in Jeff Patton's publication "User Story Mapping" \citep{pattonUserStory}. There are still a few publications about the compatibility of HCI and Agile \citep{HumanCom2:online} and Human-computer Interface expert systems for agile methods \citep{Humancom0:online}.
% 
% But none of the above differentiate between starting a new project and maintaining a product. 