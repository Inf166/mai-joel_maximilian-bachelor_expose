% chktex-file 18
% 
% Erläutern Sie in groben Zügen den jetzigen Kenntnisstand zu diesem Problem auf Basis
% der Vorgaben Ihres Industriepartners und / oder auf Basis einer Literaturrecherche
% Für Bachelor- und Masterarbeiten wird eine erste Einordnung in die Literatur erwartet:
% Was sind die wichtigsten Aussagen im Kontext Ihres Problems.
% Bitte zitieren Sie korrekt.
\addcontentsline{toc}{section}{\normalfont{Status Quo}}
\section*{Status Quo}

Although SCRUM is a wrapper for an extreme approach to agile development, it is often thought of as being the same as agile. It is used interchangeably. The reasons for this are plenty. For example, SCRUM is the most popular agile framework and therefore often used as an example for agile project management. In Addition, SCRUM predates the agile manifesto, which might also add to the confusion. At first, this does not seem like a big problem, but the reasons and ways to adopt agile and scrum are different. First, a company should understand agile and then decide on what framework to choose, not the other way around. 

SCRUM seems to be the most popular framework simply by its simplicity. At its core SCRUM promises to only use three roles, produce three artifacts and add only three new meetings. This is just a barebone framework, used as a wrapper to manage the complexity of the value stream of any company. It promises to work as a wrapper around any existing practices. It promises to improve the quality and efficiency of the product and value stream. All these promises and the simplicity of the framework make SCRUM highly attractive to a company that wants to improve but not change too much, more or less add on top of the processes they hold.

The Agile Manifesto was written by 17 software development thought leaders. It states 4 values and 12 principles, which makes it a rather abstract and short document. At its core, it is about better ways of developing software by doing it and helping others do it. Working agile means embracing collaboration, individuals, interaction, working software and responding to change. Although they still found that they also value processes, tools, documentation, contract negotiations and following a plan. But less than the things stated before.

One of the core principles of the Agile Manifesto states that they welcome changing requirements, even late in development. In addition, it also states that the best architectures, requirements, and designs emerge from self-organizing teams.

An agile business leads to the impression of chaotic disorder. But it's not true. Everybody talks to each other, there is no wasting time waiting until a meeting is planned. Clients appear on site to test and answer questions of developers and designers. Post-its are clustered on the wall or on virtual whiteboards. 

SCRUM, on the other hand, can look the same but keeps the organization structure. Meetings around the SCRUM Teams stay untouched and are kept in the waterfall manner.

Therefore SCRUM disregards two of the twelve key principles. SCRUM does not have changing requirements during a running sprint and it does not allow for self-organizing teams if put onto a team by management. If SCRUM is emerging through motivated self-organizing teams, it might work. 