% chktex-file 18
% 
\addcontentsline{toc}{section}{\normalfont{Acronyms}}
\section*{Acronyms}

\newacronym{HCD}{Human-Centered Design}

\newacronym{HCI}{Human-Computer Interaction}

\newacronym{UI}{User Interface}

\newacronym{UX}{User Experience}

\addcontentsline{toc}{section}{\normalfont{Glossary}}
\section*{Glossary}

\subsection*{Roles of SCRUM}

\newglossaryentry{Client}{Clients are the Contractors of a product. They are also known as business owners/ product owners.}

\newglossaryentry{Customer}{Customers are using the product. They are also known as shoppers or users.}

\newglossaryentry{Project Manager}{The project manager has the goal of ensuring that there are clearly defined project objectives and parameters and that the project team meets these objectives.}

\newglossaryentry{Product Manager}{The role of a product manager is to ensure product success throughout the entire product life cycle, during and beyond the initial project.}

\newglossaryentry{Agile Integrator}{The agile integrator has the task to implement agile methods into the organization and processes. He is also known as SCRUM Master. He is responsible for creating the best possible working conditions for the team and supports them in organizing themselves.}

\newglossaryentry{Desinger}{Can be a Frontend Developer, UX Designer, UI Designer or else. Has the focus on the layout of the product.}

\newglossaryentry{Architect}{Can be a Requirement Engineer, Software Engineer, Software Architect, UX Designer or else. Has the focus on the structure and dependencies of the product}

\newglossaryentry{Developer}{Can be a Frontend, Backend Developer, Software Architect or Software Engineer or else. Has the focus on the technical constraints of the product.}

\newglossaryentry{Tester}{Can be partially automated. But every piece of software that is critical to the system has to be tested, by someone or something. Usually, Quality Assurance Department writes tests and implements this testing infrastructure.}

\subsection*{Definitions}

\newglossaryentry{Requirement Engineer}{The role requirement engineer is almost completely integrated into the product owner. But due to the product owner being the client, the requirement engineer is an intelligent tool to use to refine client-side requirements. They help understand the client's needs and translate them for the team members}

\newglossaryentry{UI Designer}{A User Interface Designer usually is more focused on the visual appeal than the effectiveness of the said interface.}

\newglossaryentry{UX Designer}{A User Experience Desinger is a skilled researcher. The main focus of his work lies on the efficiency and effectiveness with the user interacts with his interface.}

\newglossaryentry{Software Architects}{Software Architects are also known as Software Engineers. It is their task to develop the concepts and constructs for the developers to build. They have the bigger picture in mind and hold parts of the vision of the final product.}

\newglossaryentry{Software Developer}{A Software Developer develops the software based on requirements given to him. He is responsible for the code he writes. Therefore Developers are usually heavily invested in writing code that is efficient, clean and robust.}

\newglossaryentry{Frontend Developer}{A Frontend Developer lays his focus on the front end of a software product. He usually works closely with UI and UX designers.}

\newglossaryentry{Backend Developer}{A Backend Developer is responsible for the frontend components to work together. He usually has the bigger picture in mind and therefore works closely with the architect together.}

\newglossaryentry{Continuous Learning}{Continous Learning describes the culture of an agency or business to always try to learn new things and refactor their processes to become better at what they are doing. Continuous Learning is an important step to becoming agile.}

\subsection*{Phases of a Product}

\newglossaryentry{Continuous Evolution}{Continous Evolution is the name for a software product life cycle. It describes its life cycle more biologically. Software products nowadays evolve. Maybe even becoming something that was not part of its vision in the beginning. See Facebook for example.}

\newglossaryentry{Continuous Exploration}{Continous Exploration is the first part of a product. The development team makes hypotheses about the product and tests them on the live system with real users.}

\newglossaryentry{Continuous Integration}{While the product is out there. The new feature becomes integrated. That's called Continuous integration.}

\newglossaryentry{Continuous Deployment}{While the team focuses on developing, testing and hypothesizing about the product. The deployment of the product should be mostly automated and therefore continous deploy parts of the product.}

\newglossaryentry{Continous Relasing}{Also known as Release on demand. In case of release on demand, the business decides at what time the product and its new features see the light of the day. But in case of continous releasing, the system always releases if something new is deployable. That way, new data streams into the next cycle and no time is wasted.}

