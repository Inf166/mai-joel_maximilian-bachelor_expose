% chktex-file 18
% 
% Human-computer interaction, agile, Lean Startup, Agile Experience Design, agile project management, Human-centered design, Toolbox
\newcommentblock{
% Problems:
The Waterfall process comes with risks for companies. To implement such a process correctly, all requirements for the finished product have to be sorted out at the beginning of the project, therefore requiring a lot of time at the start, leaving little time to react to changes. But it is almost standard for software projects to have changing requirements during the development phase, which results in undesirable products for the end-user or unsatisfied clients\cite[p. 14]{ratcliffe2011agile}. The cost of change compounds with time\cite[p. 22]{ratcliffe2011agile}.
% - Waterfall processes come with risks for companies because they have to be sure they got all requirements sorted out
%   - If not, they can not react fast enough to changes in requirements and therefore lose money and time.
%     - But it is almost given, that changes will come during development
%       - Therefore those products will more often than not fail in the real market

Out of these problems arise a few requirements to the product, that has to be met always. The product has to be desirable to the end-user. It has to be technically feasible. And to project has to deliver business value fast.
% - The project needs to be desirable to the end-user
% - The project needs to be technically feasible
% - The project needs to deliver business value fast

To solve these problems, developers and designers came up with solutions, such as working agile or having iterative design phases. Starting with agile, the most frequent problem with agile comes with implementing an agile framework in the company. All members of the project and the client have to commit to the agile manifesto. But most clients cannot leave someone at the agency to stay in contact and collaborate with the teams. Therefore a big share of Web Agencies kept parts of the original waterfall phases in their processes\citep{10spanne26:online}. But because of this management decision, developers and designers cannot fully work agile, and therefore only the negative impacts of working the agile show, such as messy tickets, volatility errors, and opaque workflows. This needs additional tweaking such as the backlog\citep{nakao2014using}.
% - Agile problems
% - All members of the project and the Client had to commit to the agile manifest
%   - A big share of all Web Agencies kept parts of the original waterfall phases in their processes
%     - Developers and Designers can not fully work with agile
%       - Only the negative Impact of agile is shown in those agencies

Improvements to Development or Design phases kept their focus on each separate phase in dictating the other\cite[p. 22]{ratcliffe2011agile}. For example, agile frameworks usually need designers to work in smaller increments\citep{Theartof49:online}, with not enough time to research, which results in concepts that do not grab the whole aspect of all requirements or rely on their assumptions\citep{HowtoCom22:online}. This results often in less collaboration between the two teams\cite[p. iv]{ratcliffe2011agile}\cite[p. 115]{Bringing52:online}. Both teams work after one another instead of parallel or intertwined\cite[p. 9]{Forbrig2015ManagingTA}. Therefore Designers are driven by the deadlines set by the project manager for their concepts and not collaborating closely with software architects leads to unviable solutions\cite[p. 4]{ratcliffe2011agile}. But Developers build working software, which is no longer valuable to the end-user because they put technical feasibility over the set boundaries by the UX team. The lack of focus put on the end-user in agile development is a major concern and the biggest weakness\cite[p. 312]{Humancom0:online}.
% - Improvements to Development or Design phases kept their focus on each separate phase in dictating the other 
% - For example Agile wants Designers to work in smaller increments, but they usually want to build a finished product knowing all requirements
%   - Instead of collaborating more, these improvements result in less collaboration
%     - Both Teams work after each other with a slight head start 
%     - Instead of working parallel or intertwined    
%   - Designers are driven by deadlines and collaborate less with developers, which leads to not viable solutions 
%   - Developers build working software, which is no longer valuable to the end-user

The Human-centered Design approach solves many of the problems above, but also comes with problems. First of all, it is important, that the Design includes all humans in the required research. Concentrating only on the end-user will leave gaps in the requirements that can lead to an undesirable product by the client. More often than not, the Researcher writes User Stories wrong\citep{userstor59:online}. They already have a solution in mind, leaving out the possibility of a better solution. Usually, the User is kept in mind during the development of the concept, by Empathy Maps or other Tools. But once the concept is finished and given to the development team, those end-user requirements are forgotten and replaced by technical restraints, leading again to an undesirable product\cite[p. 19, p. 33]{ratcliffe2011agile}.
% - Human-centered Design problems
%   - Human-centered instead of User-centered
%   - keeps the end-user in mind till a finished product concept is reached
%     - Write Business Stories instead of User Stories
%       - The way of writing a user story can prohibit the actual solution that may be more feasible
%     - During the Development of said concept, the end-user requirements are forgotten and replaced by technical restrictions

Once a product reaches the Quality Assurance Department, other problems show up, like that the final deployed product only receives irregular Updates. Neither agile frameworks nor human-centered design keeps in mind, that requirements after deployment can change, and therefore the view on sustainable products has to change too\cite[p. 18, pp. 30-31]{ratcliffe2011agile}.
% - Quality Assurance Department Problems
%   - the final product gets deployed and only receives irregular updates
%   - neither agile nor a sustainable view of digital products
%   - Requirements will change, therefore the way of reaching a solution must too

It is needed to differentiate between starting a new project and maintaining a product. When starting a new project, software architects are not included during the concept phases, which leads to no feasible solution. Designers usually waste time fetching out documentation and designs, while Developers have to wait until the design is set. When maintaining a product, Designers and Developers need to collaborate more closely. They also have to be in the meetings as Consultants to discuss feasible solutions with the client to develop a satisfying product. The business also has to listen and measure the success of the product in the live environment.
% - Differentiate between starting a new project and maintaining a product problems
%   - Starting a new project
%     - Software Architects are not included in concept phases
%     - Designers waste time on fetched-out designs, and development has to wait until the design is set
%   - Maintaining a product
%     - Designers and Developers do not collaborate closely 
%     - They do not discuss feasible solutions together with the client (project management does with no information on feasibility) 

HCI and Agile share the values of user-focus and iterative cycles, which leads to the conclusion, that combining them and solving the main problems, should lead to a workflow, that solves all key problems\citep{HumanCom2:online}. When trying to fix all of the problems stated above, we still need to keep in mind the time, budget, and scope of the projects. Therefore a subgoal should be to reduce waste and unused documentation.
% - Fixing all of the problems that remain
%   - We still need to keep in mind the time, budget, and scope of the projects
%     - subgoal should also be to minimize waste and unused documentation
}

% Currently, it is State of the art to use agile management methods, such as scrum, lean start-up, or design thinking, to generate business value faster, reduce risks, increase flexibility and improve customer satisfaction. 
% 
% Therefore, there are many publications available that discuss integrating user-centered approaches into agile development\citep{PDFAnAgi97:online, HowtoCom22:online} some even going as far as to call it human-centered\citep{HumanCen72:online, Forbrig2015ManagingTA, Agilehum49:online}. The difference is, that user-centered is end-user focussed, and human-centered means, keeping the focus on all humans interacting with the system.
% 
% As for Human-computer interaction, the most talked-about strategy is human-centered design, which is similar to an agile iterative process, but it focuses on the humans that will use the final product. Agile Experience Design\citep{ratcliffe2011agile} is another strategy that tries to improve on the negatives of agile as it is.
% 
% Another approach is to integrate agile into the human-centered design by working one sprint ahead of the design team, or working parallel and closely together, thereby sharing their finished increments and improving on each other. Last but not least there is an approach where the UX Steps of Iteration are completely integrated into the sprint. By working together and not before or aside from the development team, UX becomes an important part of the sprint which results in a product that is not only technically feasible but also usable and has improved user experience\citep{Bringing52:online}.
% 
% Surrounding Human-computer interaction, the most complete set of methods and strategies as well as recommendations can be received by the DIN Standards Committee Ergonomics on their Norms\citep{dinEnIso9241:110, dinEnIso9241:210, dinEnIso9241:220}. More Information on how to write valuable user stories is found in Jeff Patton's publication "User Story Mapping"\citep{pattonUserStory}. There are still a few publications about the compatibility of HCI and Agile\citep{HumanCom2:online} and Human-computer Interface expert systems for agile methods\citep{Humancom0:online}.
% 
% But none of the above differentiate between starting a new project and maintaining a product. 

% Currently, the agile strategy is not concerned with differentiating between a new project and maintaining a product. It is the goal of this thesis to improve on all phases of the processes. The result will take the minimal-waste aspects of lean-startup, human-centered in phases of design and development due to Human-computer interaction methods, and make faster business value through the splitting of workload into smaller, more manageable chunks, following the scrum process. For example, using the Software Architects early on, during the concept phase, will result in much more feasible solutions. Combining that with the required research of HCI will result in a feasible and human-centered solution. \\
% Therefore, the main thesis is: "\thesis"\\
% \newline
% But this thesis will also answer questions like, why integrating Human-computer interaction methods is the key to improving satisfaction, why other strategies fail to bring value/satisfaction, and also why software architects should be used early on in a project? To answer these questions, it is needed to discuss other existing strategies, analyze studies and develop a toolbox with methods and strategies for most use cases. At the end of this thesis, there will be a recommendation to new and old agencies, that want to improve their workflow, to become faster reacting and ready for a change in requirements, while still being able to satisfy the expectations of their clients and end-users.

\newcommentblock{
A new project is started

A product needs an update

A Bug is found in a product 

A Client wants a new feature

A product is transferred over to the agency
}
% The first step will be to research the available agile frameworks, keeping the focus on the established and mainly used in common working environments. Thereafter, current solutions to the key problems will be analyzed and rated. Following this, it is to decide if further methods are needed, or if the existing ones just have to be modified to accommodate a feasible workflow. All listed methods and strategies will be compared in their benefits and differentiated due to their disadvantages. Next, the decisions have to be made if the set goal of this thesis can be accomplished. Furthermore, the refined methods/tools will be recommended and enhanced by a refined vocabulary based on expert opinions. This toolbox will be additionally added with notes, which are useful for new projects or maintaining an existing product. At last, it is discussed if all gaps were successfully closed, and which are remaining, leading to follow-up projects.

\newcommentblock{
User Stories vs Business Stories:
User stories usually indicate a solution
Wrong approach:
Better approach would be the technology independent of companies, for example, the real problem of a user -> This enables much more far more solutions and more value for the company

Epics, Features, Stories:

Atomic Design:

Project Managers:
project manager has two primary goals: ensuring that there are clearly defined project objectives and parameters and that the project team meets the objectives.

Product Managers:
The role of a product manager is to ensure product success throughout the entire product life cycle, during and beyond the initial project.

Designers:\\
User Interface Designer interaction design, and usability underlie much of what we do in experience design today. They in turn were all influenced by Human-computer interaction (HCI). The focus was primarily to design for and measure task-interaction completion, efficiency, error prevention/recovery, and intuitiveness.

User Expierence Designer (xd) is concerned with the design of customers’ inter- actions with products, services, and systems that shape and influence consumer behaviour. Experience design can often involve the creation of expe- riences that traverses online and offline throughout the entire customer life cycle, and is not limited to software design. Therefore, an experience designer is often polyskilled in user-centred research, interaction and information design, and usability.

ui developers and front-end developers are closer to developers than design- ers as they spend their time writing presentation layer code. They are often the link between the front end and designers and the back end and develop- ers as they can speak both languages.

information architecture/architects categorise data or information into a coherent structure that customers can make sense of and find what they are looking for.

visual design/designer is concerned with the look and feel and the brand elements of a user interface and is most closely related to graphic design. A visual designer considers typography, colour, iconography, layout, image style, and brand development.

design research/researchers combine disciplines such as usability, ethnography, anthropology, and market research. This combined approach is used to investi- gate factors that might influence and contribute to the design of a user interface. A design researcher who specialises in research, may, in fact, do no design at all.

Developer:
Software Architects
Software Designer/System Designers
Software Engineer
Software Developer

Lean Philosophy:\\
Its about minimising waste. Using automation. Using just-in-time/real-time production (using only what you know right now).

Continuous Evolution: 
Products don't start their live after the project finishes.\\
They start it at the moment of creation and evolve over time.
(Familier with continuous design, development, delivery or evolutionary design (design during development))
By monitoring and measuring the customer experience, acting on the real-time data, continuously designing, and continuously delivering to the live environment, we truly can exist in a post-project world of continuous evolution

continuous exploration
continuous integration
continuous deployment
continuous releasing
continuous communication between developers and customers

design thinking appeared on the scene a few years ago and takes the cre- ative approach of traditional design into a business context. The idea is to enable nontraditional design people to use the creative methods of designers to inspire original thinking, more creative strategic approaches, and better product innovation. Design thinking has some great proponents, including organisations such as IDEO and Harvard Business School. Stanford University have even gone so far as to create their own d:school:Institute of Design.
It has helped to bring creative thinking and human-centred design into the boardroom and has helped to pave the way for a better appreciation of the value of experience design.

service design is about creating ecosystems of connected products, services, communications, and environments with people and for people that enable the co-creation of value. Where user-interface design, for example, looks primarily at the user interaction for a single system, service design considers all the links in the customer-provider chain across channels, across organisational silos, across systems. It has influenced how experience designers think beyond the original scope of the website and think more holistically about the customer experience and how the digital channels integrate with the offline channels.

Lean start-up applies the philosophies of just-in-time and just-enough to the business start-up process. It looks to rapidly generate and test business ideas in the marketplace at lightning speed. The idea is to either fail fast— that is, kill the concept that isn’t working as quickly as possible, before wast- ing any time or money developing it further, or scale quick—so for concepts that have shown promise of success in early testing we want to get the “mini- mal viable product” into the hands of customers so we can test and learn.
Of course experience design has always been about testing early and testing often, but lean start-up begins by testing the business case.

smaller chunks of work = faster business value / more frequently
}