% chktex-file 18
\addcontentsline{toc}{section}{\normalfont{Research}}
\section*{Research}

\subsection{User Story considered harmful}
User stories usually indicate a solution

Wrong approach:
Better approach would be the technology independent of companies, for example, the real problem of a user -> This enables much more far more solutions and more value for the company
\citep{userstor59:online}

\subsection{The digitization podcast}
Complex world.
Today like this, tomorrow like that.
Best case strategies and
Linear management/business models are not suitable

Because nobody wants something new => difficult to overcome the direct rejection of management
Constantly reinventing companies
Show advantages for everyone in the company
Do not judge afterwards but participate more
First start with yourself and before life until the middle and superior management watches


\subsection{Combining Agile and HCD}
Agile Development Sprint Board Backlog:
Expand every ticket to a column with the customer/user's feedback on how many have worked on it and how high the prio is
In addition, the classification of the person tests (profession, age etc.)
Every sprint review show/ make/ make known among developers because they prefer to simply implement solutions
\citep{nakao2014using}

\subsection{Agile Experience Design}
"The agile movement must continue to innovate and adapt to remain relevant" \citep{ratcliffe2011agile:fw}

Key Problem: "But, as it turns out, specialisation wasn't the primary problem — collaboration was" \citep{ratcliffe2011agile:fw}

"Design is currently driven by deadlines" \citep{ratcliffe2011agile:4}

"Invariably, a business dreams up a vision for a product or service and sets expectations with stakeholders and shareholders. Then the design team spends a heap of time and money on detailed design, without necessarily determining if the concept is feasible." \citep{ratcliffe2011agile:6}

Agile Experience Design Manifesto: 
people over process, 
working software over documentation, 
collaboration over contract negotiation, 
and responding to change over sticking to a plan.
The Agile Manifesto: 
individuals and interactions - over processes and tools, 
Working software - over comprehensive documentation, 
customer collaboration - over contract negotiation, 
responding to change - over following a plan.
\citep{theagilemanifesto:online}

"These agile guys even write their software requirements as user stories which state who the user is and what his goals are, so the business trusts that IT have their customers in mind." \citep{ratcliffe2011agile:7}

"The aim of agile is to get to code as quickly as possible." So if you include design in this process, you will probably add to the business value. \citep{ratcliffe2011agile:9}

"Agile experience design is:
inclusive - rather than elitist
emergent with direction - rather than up front
integrated and collaborative - rather than handed over the fence
considerate of customer, business, and technology needs - rather than biased toward a single factor" \citep{ratcliffe2011agile:9}

Concurrent project is better than squential ones?

Key Problem: "For an agile project to have a good chance of success there must be both understanding and commitment, not only from the immediate project team, but also from the wider organisation." \citep{ratcliffe2011agile:11}

Solution: "A cross-functional senior management team within the organisation needs to sponsor the adoption of agile, either for a particular project or as a wider pro- gramme of change. The senior management team then needs to push a mandate through the divisional lines to the functional managers where the team members come from." \citep{ratcliffe2011agile:11}

Goal: "The product needs to deliver business value, be technically feasible, and needs to be desirable by the end customer. Creating a desirable product does not happen by accident" \citep{ratcliffe2011agile:11}

Enablers are design thinking, service design, product design, graphic design, user-experience design\\
Reapply them in an agile project environment \citep{ratcliffe2011agile:12}
Lean manufacturing and lean startup as starters

Solution: "avoid creating 'waste' or stockpiling anything that could become obsolete with change" \citep{ratcliffe2011agile:12} 

Designers and Developers and Managers work closly together
Desigerns work ahead of developers and test with customers during sprints.
While the product is being built and the design is being enhanced or getting more specified.
"Get out of the office and onto the streets to get feedback from the target audience." \citep{ratcliffe2011agile:13}

Solution: "When a digital product or service is launched, it’s the start of a new understanding of how customers interact and how they feel about the product. The live environment is the perfect place to measure and analyse how the digital customer experience is performing. With the right tools it’s possible to quickly assess what’s working and what’s not. Then by using techniques such as A/B testing and multivariate testing designers can experiment with design variations, and use the results to optimise the digital customer experience." \citep{ratcliffe2011agile:14}

"The traditional design process is outdated and isn’t the most efficient or effec- tive path to success." \citep{ratcliffe2011agile:14}
"By thinking about product improvements in a different way, and coupling continuous design with continuous delivery, we can reap the benefits of real-time continuous improvement." \citep{ratcliffe2011agile:14}


\textbf{waterfall project management and illustrate where problems lie}
"Waterfall comprises a series of discrete phases. A phase begins only after the agreed deliverables from the previous phase are completed to the agreed standard and are accepted and signed off by project governance." \citep{ratcliffe2011agile:18}

"By definition, done means the task of doing an activity has ended and the job is complete." \citep{ratcliffe2011agile:18}
Print: 
"Once the project goes to print it is considered done." \citep{ratcliffe2011agile:18}
Digital Waterfall: 
"The completed specification is thrown over the wall for developers while the designers head to the pub to celebrate a job well done." \citep{ratcliffe2011agile:19}
"development is considered done when the code is complete and has been sent to quality assurance (QA) for testing." \citep{ratcliffe2011agile:19}
"overall project is considered done when everything that was requested in the requirements document is built, tested, deployed, and proven to work cohesively and without fault in a live environment" \citep{ratcliffe2011agile:19}

Fuck the highlevel design > Show Results early in development

Design is limited by the research team translated business requirements.
Development is limited by designs made design, which should describe the design rationale and detail.
That causes develepment with the task to interpret the design detail, so its like it was originally intended, while also wrangle with the technical constraints. 
\textbf{The cost of change compounds with time}
Because of that, it is understandable that designers have to be able to concentrate on their design, so they make no mistakes which cost much more effort to correct during the build phase.
\citep{ratcliffe2011agile:22}

Working agile means taking smaller workloads and using concepts such as iterative development. This results in quicker delivery of smaller pieces which intern results in earlier value and enables testing to accour more often. \citep{ratcliffe2011agile:25} 

"The intention of delivering value can be derailed by politics around the correct processes and the appropriate tools."
"Valuing individuals and interactions means project team members having lots of opportunity to verbally communicate and collaborate throughout the project life cycle"
"Why do designers prefer to work in their private space?" "We get protective" They don't want to be juged.
"working closely with team members has a different result"
\citep{ratcliffe2011agile:25}

"the designers couldn’t seem to grasp the complex interactions and dependen- cies within the process."
"designers spend a week at the client site and do the designs in their offices. That way IT would be able to see how the designs were progressing as they were developed and ensure that issues could be addressed as soon as possible" \citep{ratcliffe2011agile:26}

"the agile team are surrounded by stuff on the walls: cards, sticky notes, diagrams on whiteboards. The waterfall team works from documents." \citep{ratcliffe2011agile:27}

Making documentation look good should be an afterthought, because ones singed off, the documentation cannot be changed without making more work. But because changes are inevitable, this strategy results in loss of time. Working software or designs should be valued more then "pretty" presentations and documentation. \citep{ratcliffe2011agile:29}
A more reasonable approach should be to only do documentation which is really used or needed by teammates or customers.
"The QA testers need comprehensive documenta- tion from the designer to ensure that what development has built is what was intended." \citep{ratcliffe2011agile:29}
"The test, or acceptance criteria, emerges alongside the design in the user-story detail. In this way the user story and the accompanying wireframe are all the documentation that is needed to create working software." \citep{ratcliffe2011agile:29}

Waterfall:
"It is expected that the customer knows exactly what he needs and wants before any work starts. It is expected (contrary to experience) that require- ments will not change." Debates about what was agreed on arise. "The root cause of the issue is an inflexible philosophy that is not designed to accommodate change." \citep{ratcliffe2011agile:30}
"business coming up with requirements and then IT doing the estimation, planning, and development"

Agile:
Project Management in three dimensions: time, scope, and budget
"We acknowledge that we can’t know everything up front and we accept that aspects of the project will change." 
"IT and the business drive and prioritise requirements, which are then estimated" "create a plan for what can be delivered"
\citep{ratcliffe2011agile:30-31}

The Contractor is from now on called client or business owner/ product owner.
Customers of the software product are called shoppers or users.

"we'd argue it is far more valuable to iterate your designs on a live product with real feedback than to try to polish something in the lab or rely on research." \citep{ratcliffe2011agile:32}
Your plan is based on what you know right now. All attachment is suffering. Keep your plan flexible so changes make the project not impossible. This results in a better product.

"While we agree there is value in working software, we believe there is little value if the working soft- ware doesn’t build the right product for the right people at the right time." \citep{ratcliffe2011agile:33}
A designer and a developer working together with the client. Developer and client define what the product should do and what datasets arise. Designer gets the defined datasets and uses the data he has access to, based on the sets he gets.\\
Theirfore a developer knows exactly how to store the data and how it can be modified and how modifing the data can result in side effects.

Jeff Patton agileproductdesign.com -> Agile means smaller chunks result not in a cohesive product

Agile:
someone in the business has an idea.
the development team decompose the idea into stories.
designers say what colours to use.
the team deliver in increments, tackling the technically hardest functionality first.
hey deliver high- quality stories on time and on budget, adding functionality iteratively.
yet the end result is not quite what everyone expected.
\citep{ratcliffe2011agile:34-35}

Agile with User Experience:
someone in the business sees an opportunity.
Key Difference: a cross-functional team comes together.
everyone has lots of different ideas.
everyone collaborates to reach a shared understanding.
they discover new ideas by getting out of the office.
Key Difference: rapid sketching brings the vision to life.
Key Difference: with a clearer vision we produce, prioritise, and estimate user stories and create a development plan.
Key Difference: detail emerges throughout the development process, directed by the vision.
the result is a useful, usable, and desirable product.
\citep{ratcliffe2011agile:34-35}

"Design is Already Agile" 
"Experience design has always been about collaborating with end customers." 
"We need to spread the design effort over the length of the project, rather than trying to do it all up front."
"We need to learn how to collaborate not only with the end customers, but also with the other members of the team to produce the design."
\citep{ratcliffe2011agile:36}

Continuous Evolution:\\
Products don't start their live after the project finishes.
They start it at the moment of creation and evolve over time.\\
Lean Philosophy:\\
Its about minimising waste. Using automation. Using just-in-time/real-time production (using only what you know right now).\\
continuously listen, monitor, and interact with them and be prepared to make changes in line with what we know.
\citep{ratcliffe2011agile:37}

"differences between a project and a product become blurred" \citep{ratcliffe2011agile:38}

Agile leaves the waterfall thinking and splits the design phase up into indiviual short periods of creation.
"Agile has openly declared that it is opposed to big, up-front design, which sounds like a criticism of design." 
\citep{ratcliffe2011agile:43}

"software architects are no longer developers." \citep{ratcliffe2011agile:44}

"somebody on the team has to have the determination to ensure that the design quality stays high." Martin Fowler \citep{ratcliffe2011agile:45}

Extreme Programming sucks because it does not use a foundation and wants to get straigt to coding. This creates loss in time, loss in budget and waste.

even before a project is kicked off—more on that later?


\citep{ratcliffe2011agile}