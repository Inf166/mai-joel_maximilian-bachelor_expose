\addcontentsline{toc}{section}{\normalfont{Research}}
\section*{Research}

\subsubsection{User Story considered harmful}
User stories usually indicate a solution

Wrong approach:
Better approach would be the technology independent of companies, for example, the real problem of a user -> This enables much more far more solutions and more value for the company
\citep{userstor59:online}

\subsubsection{The digitization podcast}
Complex world.
Today like this, tomorrow like that.
Best case strategies and
Linear management/business models are not suitable

Because nobody wants something new => difficult to overcome the direct rejection of management
Constantly reinventing companies
Show advantages for everyone in the company
Do not judge afterwards but participate more
First start with yourself and before life until the middle and superior management watches


\subsubsection{Combining Agile and HCD}
Agile Development Sprint Board Backlog:
Expand every ticket to a column with the customer/user's feedback on how many have worked on it and how high the prio is
In addition, the classification of the person tests (profession, age etc.)
Every sprint review show/ make/ make known among developers because they prefer to simply implement solutions
\citep{nakao2014using}

\subsubsection{Agile Experience Design}
"The agile movement must continue to innovate and adapt to remain relevant" \citep{ratcliffe2011agile:fw}

Key Problem: "But, as it turns out, specialisation wasn't the primary problem — collaboration was" \citep{ratcliffe2011agile:fw}

"Design is currently driven by deadlines" \citep{ratcliffe2011agile:4}

"Invariably, a business dreams up a vision for a product or service and sets expectations with stakeholders and shareholders. Then the design team spends a heap of time and money on detailed design, without necessarily determining if the concept is feasible." \citep{ratcliffe2011agile:6}

Agile Manifesto: "people over process, working software over documentation, collaboration over contract negotiation, and responding to change over sticking to a plan."

"These agile guys even write their software requirements as user stories which state who the user is and what his goals are, so the business trusts that IT have their customers in mind." \citep{ratcliffe2011agile:7}

"The aim of agile is to get to code as quickly as possible." So if you include design in this process, you will probably add to the business value. \citep{ratcliffe2011agile:9}

"Agile experience design is:
inclusive - rather than elitist
emergent with direction - rather than up front
integrated and collaborative - rather than handed over the fence
considerate of customer, business, and technology needs - rather than biased toward a single factor" \citep{ratcliffe2011agile:9}

Concurrent project is better than squential ones?

Key Problem: "For an agile project to have a good chance of success there must be both understanding and commitment, not only from the immediate project team, but also from the wider organisation." \citep{ratcliffe2011agile:11}

Solution: "A cross-functional senior management team within the organisation needs to sponsor the adoption of agile, either for a particular project or as a wider pro- gramme of change. The senior management team then needs to push a mandate through the divisional lines to the functional managers where the team members come from." \citep{ratcliffe2011agile:11}

Goal: "The product needs to deliver business value, be technically feasible, and needs to be desirable by the end customer. Creating a desirable product does not happen by accident" \citep{ratcliffe2011agile:11}

Enablers are design thinking, service design, product design, graphic design, user-experience design\\
Reapply them in an agile project environment \citep{ratcliffe2011agile:12}
Lean manufacturing and lean startup as starters

Solution: "avoid creating 'waste' or stockpiling anything that could become obsolete with change" \citep{ratcliffe2011agile:12} 

Designers and Developers and Managers work closly together
Desigerns work ahead of developers and test with customers during sprints.
While the product is being built and the design is being enhanced or getting more specified.
"Get out of the office and onto the streets to get feedback from the target audience." \citep{ratcliffe2011agile:13}

Solution: "When a digital product or service is launched, it’s the start of a new understanding of how customers interact and how they feel about the product. The live environment is the perfect place to measure and analyse how the digital customer experience is performing. With the right tools it’s possible to quickly assess what’s working and what’s not. Then by using techniques such as A/B testing and multivariate testing designers can experiment with design variations, and use the results to optimise the digital customer experience." \citep{ratcliffe2011agile:14}

"The traditional design process is outdated and isn’t the most efficient or effec- tive path to success." \citep{ratcliffe2011agile:14}
"By thinking about product improvements in a different way, and coupling continuous design with continuous delivery, we can reap the benefits of real-time continuous improvement." \citep{ratcliffe2011agile:14}


\textbf{waterfall project management and illustrate where problems lie}
"Waterfall comprises a series of discrete phases. A phase begins only after the agreed deliverables from the previous phase are completed to the agreed standard and are accepted and signed off by project governance." \citep{ratcliffe2011agile:18}

"By definition, done means the task of doing an activity has ended and the job is complete." \citep{ratcliffe2011agile:18}
Print: 
"Once the project goes to print it is considered done." \citep{ratcliffe2011agile:18}
Digital Waterfall: 
"The completed specification is thrown over the wall for developers while the designers head to the pub to celebrate a job well done." \citep{ratcliffe2011agile:19}
"development is considered done when the code is complete and has been sent to quality assurance (QA) for testing." \citep{ratcliffe2011agile:19}
"overall project is considered done when everything that was requested in the requirements document is built, tested, deployed, and proven to work cohesively and without fault in a live environment" \citep{ratcliffe2011agile:19}

Fuck the highlevel design > Show Results early in development


\citep{ratcliffe2011agile}