% chktex-file 8
Clients (Product Owners):
The Contractor is from now on called client or business owner/ product owner.

Users/Shoppers:
Customers of the software product are called shoppers or users.

User Stories vs Business Stories:
User stories usually indicate a solution
Wrong approach:
Better approach would be the technology independent of companies, for example, the real problem of a user -> This enables much more far more solutions and more value for the company

Epics, Features, Stories:

Atomic Design:

Project Managers:
project manager has two primary goals: ensuring that there are clearly defined project objectives and parameters and that the project team meets the objectives.

Product Managers:
The role of a product manager is to ensure product success throughout the entire product life cycle, during and beyond the initial project.

Designers:\\
User Interface Designer interaction design, and usability underlie much of what we do in experience design today. They in turn were all influenced by Human-computer interaction (HCI). The focus was primarily to design for and measure task-interaction completion, efficiency, error prevention/recovery, and intuitiveness.

User Expierence Designer (xd) is concerned with the design of customers’ inter- actions with products, services, and systems that shape and influence consumer behaviour. Experience design can often involve the creation of expe- riences that traverses online and offline throughout the entire customer life cycle, and is not limited to software design. Therefore, an experience designer is often polyskilled in user-centred research, interaction and information design, and usability.

ui developers and front-end developers are closer to developers than design- ers as they spend their time writing presentation layer code. They are often the link between the front end and designers and the back end and develop- ers as they can speak both languages.

information architecture/architects categorise data or information into a coherent structure that customers can make sense of and find what they are looking for.

visual design/designer is concerned with the look and feel and the brand elements of a user interface and is most closely related to graphic design. A visual designer considers typography, colour, iconography, layout, image style, and brand development.

design research/researchers combine disciplines such as usability, ethnography, anthropology, and market research. This combined approach is used to investi- gate factors that might influence and contribute to the design of a user interface. A design researcher who specialises in research, may, in fact, do no design at all.

Developer:
Software Architects
Software Designer/System Designers
Software Engineer
Software Developer

Lean Philosophy:\\
Its about minimising waste. Using automation. Using just-in-time/real-time production (using only what you know right now).

Continuous Evolution: 
Products don't start their live after the project finishes.\\
They start it at the moment of creation and evolve over time.
(Familier with continuous design, development, delivery or evolutionary design (design during development))
By monitoring and measuring the customer experience, acting on the real-time data, continuously designing, and continuously delivering to the live environment, we truly can exist in a post-project world of continuous evolution

continuous exploration
continuous integration
continuous deployment
continuous communication between developers and customers

design thinking appeared on the scene a few years ago and takes the cre- ative approach of traditional design into a business context. The idea is to enable nontraditional design people to use the creative methods of designers to inspire original thinking, more creative strategic approaches, and better product innovation. Design thinking has some great proponents, including organisations such as IDEO and Harvard Business School. Stanford University have even gone so far as to create their own d:school:Institute of Design.
It has helped to bring creative thinking and human-centred design into the boardroom and has helped to pave the way for a better appreciation of the value of experience design.

service design is about creating ecosystems of connected products, services, communications, and environments with people and for people that enable the co-creation of value. Where user-interface design, for example, looks primarily at the user interaction for a single system, service design considers all the links in the customer-provider chain across channels, across organisational silos, across systems. It has influenced how experience designers think beyond the original scope of the website and think more holistically about the customer experience and how the digital channels integrate with the offline channels.

Lean start-up applies the philosophies of just-in-time and just-enough to the business start-up process. It looks to rapidly generate and test business ideas in the marketplace at lightning speed. The idea is to either fail fast— that is, kill the concept that isn’t working as quickly as possible, before wast- ing any time or money developing it further, or scale quick—so for concepts that have shown promise of success in early testing we want to get the “mini- mal viable product” into the hands of customers so we can test and learn.
Of course experience design has always been about testing early and testing often, but lean start-up begins by testing the business case.

smaller chunks of work = faster business value / more frequently