% chktex-file 18
% Hier geht es nicht darum, Ihre persönliche Motivation zu beschreiben.
% Stattdessen sollten Sie erörtern, welche Lücke der Status Quo hinsichtlich Ihrer Problemstellung lässt.
% Formulieren Sie hier Ihre Forschungsfrage / Ihre Design Challenge / Ihre Hypothese.
% Falls es Unterpunkte zu einer Hauptfrage gibt, formulieren Sie maximal drei.
% Die Forschungsfrage kann sich verändern, wenn Sie mehr Kenntnisse im Kontext Ihrer Arbeit gewinnen.
% Bitte beachten Sie, dass die Forschungsfrage für Ihren Zeitrahmen angemessen ist.
\addcontentsline{toc}{section}{\normalfont{Derived Thesis}}
\section*{Derived Thesis}

Currently, the agile strategy is not concerned with differentiating between a new project and maintaining a product. It is the goal of this thesis to improve on all phases of the processes. The result will take the minimal-waste aspects of lean-startup, human-centered in phases of design and development due to Human-Computer-Interaction methods, and make faster business value through the splitting of workload into smaller, more manageable chunks, following the scrum process. For example, using the Software Architects early on, during the concept phase, will result in much more feasible solutions. Combining that with the required research of HCI will result in a feasible and human-centered solution. \\
Therefore, the main thesis is: "Integrating Human-Computer-Interaction methods into agile software development while keeping waste at a minimum."\\
\newline
But this thesis will also answer questions like, why integrating human-computer interaction methods is the key to improving satisfaction, why other strategies fail to bring value/satisfaction, and also why software architects should be used early on in a project? To answer these questions, it is needed to discuss other existing strategies, analyze studies and develop a toolbox with methods and strategies for most use cases. At the end of this thesis, there will be a recommendation to new and old agencies, that want to improve their workflow, to become faster reacting and ready for a change in requirements, while still being able to satisfy the expectations of their clients and end-users.